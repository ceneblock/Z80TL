\documentclass[letterpaper, 12pt]{article}
\begin{document}
\author{Eric Neblock}
\title{Trivial Language for Z80}
\date{Last updated: \today}
\maketitle
\section{Introduction}
This document will give a brief overview of how the \textit{\underline{Trivial
  Language for Z80}} works and how to actually program in it.
program in it. 
\section{Reserve Words and Keywords}
There are several reserve words within the language. They are as follows:
\subsection{Reserve Words}
The folling list is to demostrate which words can not be used in the language
other than their intended purpose.
\begin{enumerate}
\item main()
\item send\_int(ADDRESS, int)
\item send\_char(ADDRESS,char)
\item send\_int\_array(ADDRESS, int[], size)
\item send\_char\_array(ADDRESS, char[], size)
\item read\_int(ADDRESS, int)
\item read\_char(ADDRESS, char)
\item read\_int\_array(ADDRESS, int[], size)
\item read\_char\_array(ADDRESS, char[], size)
\item for
\item while
\item return
\item function
\end{enumerate}
\subsection{Data Types}
The following are data types that exist in the language. These can not be
overridden or expanded.
\begin{enumerate}
\item int
\item char
\item float
\item array $\rightarrow$ char[constant]
\end{enumerate}

Variables must start with a character (case doesn't matter) and
\textbf{\underline{can not}} start with an \_.

Identifiers are case insensitive within their scope. So, if you have a variable
called \textit{var} in main, then it can be referred to \textit{var, Var, VAR,
  vAr, and so on}.

Each function creates a new scope, leaving you the option to have variables
called the same thing and have different meaning in functions (\textit{main() can
    have ``char x" and fun1() can have ``int x"}) 
\subsection{Keywords}
The following are data types that deal with interacting with the data bus.
\begin{enumerate}
\item ADDRESS
\item WRITE
\item READ
\end{enumerate}
\subsection{Comments}
Our language has only one way to enter comments and that is in C-style
(\textit{/* comment */})
\section{Syntax}
In this language, we have certain rules on how to have things up and running.

For example, there can only be one statement per line -- with some exceptions.
Blank lines are ignored. The assembly will put comments at the start of a block,
so:
\newline
\texttt{int x = 0 /* set a variable */}
will turn into:
\newline
\newline
\texttt{x defb 0 ; set a variable }
\section{Sample Programs}
\texttt{ 
main()
\newline
$\{$
\newline
  char type[11]
\newline
  type = ``hello world"
\newline
  send\_char\_array(0xFF00, type, size)
\newline
$\}$
\newline
\newline
function fun(int x)
\newline
$\{$
\newline
  x = x + 1
\newline
  return x
\newline
$\}$
\newline
main()
\newline
$\{$
\newline
  int x
\newline
  x = 11
\newline
  x = fun(x)
\newline
  send\_int(0x00FF, x)
\newline
$\}$
\newline
}
\end{document}
